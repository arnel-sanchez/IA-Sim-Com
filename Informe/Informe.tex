\documentclass[12pt, letterpaper,spanish]{article}
\usepackage[letterpaper]{geometry}
\usepackage[T1]{fontenc} %para escribir con caracteres utf8 como tildes
\usepackage[utf8]{inputenc} %para escribir con caracteres utf8 como tildes
\usepackage[spanish]{babel} %para escribir con caracteres utf8 como tildes
\usepackage{lmodern} % latin modern font
\usepackage{color}
\usepackage[svgnames]{xcolor} % para hacer enfasis con colores
\usepackage{amssymb}
\usepackage{amsmath}
\usepackage{hyperref}
\usepackage{multirow}
\usepackage{array}
\usepackage{mathtools} % para el modulo
\DeclarePairedDelimiter\abs{\lvert}{\rvert}%
\DeclarePairedDelimiter\norm{\lVert}{\rVert}%
\hypersetup{
	colorlinks=true,
	linkcolor=black,
	filecolor=magenta,      
	urlcolor=cyan,
}

\usepackage{listings}
\definecolor{background}{rgb}{0.52, 0.52, 0.51}
\definecolor{keywords}{rgb}{1.0, 0.0, 0.5}
\definecolor{comments}{rgb}{0.75, 0.75, 0.75}
\definecolor{strings}{rgb}{0.91, 0.84, 0.42}
\definecolor{names}{rgb}{0.4, 1.0, 0.0}
\definecolor{white}{rgb}{1.0, 1.0, 1.0}
\definecolor{types}{rgb}{0.13, 0.67, 0.8}
\definecolor{numbers}{rgb}{0.54, 0.17, 0.89}

\lstdefinelanguage{PSharp}{
	keywords={method, include, if, else, elif, while, continue, break, return},
	keywordstyle=\color{keywords}\bfseries,
	ndkeywords={int, void, rider, true, false},
	ndkeywordstyle=\color{types}\bfseries,
	identifierstyle=\color{white},
	sensitive=false,
	emph={print, test, example, Rossi, curves},
	emphstyle=\color{names},
	comment=[l]{//},
	morecomment=[s]{/*}{*/},
	commentstyle=\color{comments}\ttfamily,
	stringstyle=\color{strings}\ttfamily,
	morestring=[b]',
	morestring=[b]"
}

\lstset{
	language=PSharp,
	backgroundcolor=\color{background},
	extendedchars=true,
	basicstyle=\footnotesize\ttfamily,
	showstringspaces=false,
	showspaces=false,
	numbers=left,
	numberstyle=\footnotesize,
	numbersep=9pt,
	tabsize=2,
	breaklines=true,
	showtabs=false,
	captionpos=b
}


\usepackage{amsthm} % para teoremas y lemas
\usepackage{graphicx} % para imagenes
\usepackage{titling} % para el pretitle
\usepackage{clrscode3e} % para pseudocodigos
\usepackage[stable]{footmisc} % para permitir footnotes in los encabezados

%\usepackage[cache=false]{minted} %para representar los codigos necesita config extra
%\renewcommand\theFancyVerbLine{\normalsize\arabic{FancyVerbLine}} % incrementar el tamaño de linenumbers de minted


\newtheorem{theorem}{Teorema}[subsection]
\newtheorem{corollary}{Corolario}[theorem]
\newtheorem{lemma}{Lema}[subsection]
\newtheorem{lemmacorollary}{Corolario}[lemma]
\theoremstyle{definition}
\newtheorem*{definition}{Definición}
\theoremstyle{remark}
\newtheorem*{remark}{Observación}
\renewcommand{\qedsymbol}{$\blacksquare$} % se remplaza el simbolo vacio de final de demostración
\graphicspath{ {./_static/} }

%--------------------------
% NO MODIFICAR ESTE DOCUMENTO
%--------------------------
	
\begin{document}
	
\begin{titlepage}
	\begin{center}
		\includegraphics[width = 3cm]{escudoUH} 
	\end{center}
	\begin{center}
		Universidad de La Habana \\\vspace{0.2cm} Facultad de Matemática y Computación
	\end{center}
	\centering
	\vspace{1cm} \par
	{\scshape\Large Proyecto de Compilación + Intenteligencia Artificial + Simulación\par}
	\vspace{5mm} \par
	{\scshape\Huge Simulador de un Jefe Técnico de MotoGP\par}
	\vspace{5mm} \par
	\vfill
	{\Large \textbf{Autores:} \par}
	{\large Arnel Sánchez Rodríguez \space Grupo: C312 \par}
	\href{mailto:arnelsanchezrodriguez@gmail.com}{arnelsanchezrodriguez@gmail.com}\par
	\vspace{3mm} \par
	{\large Samuel Efraín Pupo Wong \space Grupo: C312 \par}
	\href{mailto:s.pupo@estudiantes.matcom.uh.cu}{s.pupo@estudiantes.matcom.uh.cu}
	\vspace{3mm} \par
	{\large Darián Ramón Mederos \space Grupo: C312 \par}
	\href{mailto:darianrm24@gmail.com}{darianrm24@gmail.com}
	\vspace{3mm} \par
	\vfill
	{\Large 2021-2022 \par}
\end{titlepage}	
\pagebreak
\tableofcontents
\pagebreak
\section{Motociclismo de Velocidad}
	El motociclismo de velocidad es una modalidad deportiva del motociclismo disputada en circuitos de carreras pavimentados. Las motocicletas que se usan pueden ser prototipos, es decir desarrolladas específicamente para competición, o derivadas de modelos de serie (en general motocicletas deportivas) con modificaciones para aumentar las prestaciones. En el primer grupo entran las que participan en el Campeonato Mundial de Motociclismo, y en el segundo las Superbikes, las Supersport y las Superstock.
	
	Las motocicletas deben presentar una serie de características como son estabilidad, alta velocidad (tanto en recta como en paso por curva), gran aceleración, gran frenada, fácil maniobrabilidad y bajo peso.
	
	\begin{center}
		\includegraphics[width = 12cm]{imagen1} 
	\end{center}	
	
\section{Estructura del Paddock}
	El \textbf{Jefe Técnico} de cada estructura se configura como una personalidad de bastante importancia dentro de un box, pues es quién se encarga de dirigir y controlar que todo funcione como un excelente engranaje que gane carreras. De igual importancia es la telemetría dentro de un box en MotoGP. Al fin y al cabo, los \textbf{Ingenieros Telemétricos} son las personas que se encargan de analizar, leer y comprender todos los datos proporcionados por el piloto, así como transmitirselos en boca al protagonista. Se trata de una figura de la que depende mucha de la información acerca de cualquier cambio realizado en la moto o asumir los puntos más fuertes de sus pilotos. Los \textbf{Mecánicos} también desempeñan un papel fundamental a la hora de construir la máquina perfecta.

\section{Pista}
	Se utilizará como referencia el circuito de Misano, Misano World Circuit Marco Simoncelli, autódromo localizado en la fracción de Santa Mónica, comuna de Misano Adriático (provincia de Rímini), región de Emilia-Romaña, Italia. 
	
	\begin{center}
		\includegraphics[width = 12cm]{circuito} 
	\end{center}
	
\section{Definición del Problema}
	Existirán varios pilotos con sus respectivas motos, las cuáles difieren entre sí en cuanto a sus prestaciones. Cada piloto posee su propio método de manejo, siendo algunos más cuidadosos y otros más agresivos. La pista se encuentra influenciada por el accionar del clima, puesto que no es lo mismo el manejo durante un día soleado que bajo la lluvia. Por tanto, el resultado de un piloto se verá condicionado por su moto, su modo de conducción y el clima.\par
	Sin embargo, durante la carrera las condiciones pueden variar y es necesario realizar los ajustes necesarios para que el piloto mejore su rendimiento, los cuales se harán al pasar de una seccion a otra de la pista o al finalizar cada vuelta. Esto podrá hacerse utilizando un lenguaje imperativo, mediante el uso de palabras claves para que el piloto no necesite analizar situaciones complejas y pueda concentrarse en pilotar de la forma mas eficiente posible.\par
	De esta manera, la simulación de la carrera será dinámica, puesto que entre las vueltas podrán existir variaciones provocadas por los ajustes propuestos por el Jefe Técnico, el cuál podrá ser una persona o una IA.\par
	
\section{Definición del Lenguaje}
	\subsection{Introducción a PSharp (P)}
	\subsubsection{Hello world!}
	
		\begin{lstlisting}[language={PSharp}, label={Hello World!}]
			method void main() {
				print("Hello World!");
			}
		\end{lstlisting}	
		Los archivos de P\# suelen tener la extensión de archivo .pys.
		
		
	\subsubsection{Estructura del Programa}
		Los conceptos organizativos clave en P\# son los programas y los miembros. Los programas P\# constan de uno o más archivos fuente. Los programas declaran miembros. Los métodos y propiedades son ejemplos de miembros.
		
	\subsubsection{Tipos y Variables}
		\begin{center}
			\begin{tabular}{| c | c | m{5cm} | }
				\hline
				Categoría & Tipo & Descripción \\ \hline
				\multirow{3}{*}{Tipos por Valor} & \multirow{3}{*}{Tipos Simples} & * Entero con signo: int \\
				 &  & * Punto flotante IEEE: double \\
				 &  & * Booleanos: bool \\ \hline
				\multirow{3}{*}{Tipos por Referencia} & Tipos que aceptan valores NULL & * Extensiones de todos los demás tipos de valor con un valor nulo \\
				 & \multirow{2}{*}{Tipos de matrices} & * Unidimensionales y multidimensionales, por ejemplo, int [] e int [,] \\
				 &  & * Cadenas Unicode: string \\ \hline
			\end{tabular}
		\end{center}
	\subsubsection{Análisis Léxico}
		\textbf{input}\par
		: input\_element* new\_line\par
		| directive\par
		;\par
		
		\textbf{input\_element}\par
		: whitespace\par
		| comment\par
		| token\par
		;\par
		
		\underline{\textbf{Terminadores de línea}}\par
		\textbf{new\_line}\par
		: '<Caracter de retorno (U+000D)>'\par
		| '<Caracter de avance de línea (U+000A)>'\par
		;\par
		
		\textbf{whitespace}\par
		: '<Cualquier personaje con clase Unicode Zs>'\par
		| '<Caracter de tabulación horizontal (U+0009)>'\par
		;\par
		
		\underline{\textbf{Comentarios}}\par
		\textbf{comment}\par
		: '\#' comment\_section '\#'\par
		;\par
		
		\underline{\textbf{Tokens}}\par
		\textbf{token}\par
		: identifier\par
		| keyword\par
		| literal\par
		| operator\_or\_punctuator\par
		;\par
		
		\underline{\textbf{Identificadores}}\par
		\textbf{identifier}\par
		: '<Un identificador que no es una palabra clave>'\par
		| identifier\_start\_character identifier\_part\_character*\par
		;\par
		
		\textbf{identifier\_start\_character}\par
		: letter\_character\par
		| '<Caracter guión bajo (U+005F)>'\par
		;\par
		
		\textbf{identifier\_part\_character}\par
		: letter\_character\par
		| decimal\_digit\par
		| '<Caracter guión bajo (U+005F)>'\par
		;\par
		
		\textbf{letter\_character}\par
		: uppercase\_letter\_character\par
		| lowercase\_letter\_character\par
		;\par
		
		\textbf{uppercase\_letter\_character}\par
		: 'A' | 'B' | 'C' | 'D' | 'E' | 'F' | 'G' | 'H'\par 
		| 'I' | 'K' | 'L' | 'M' | 'N' | 'O' | 'P' | 'Q'\par 
		| 'R' | 'S' | 'T' | 'V' | 'X' | 'Y' | 'Z'\par    
		;\par
		
		\textbf{lowercase\_letter\_character}\par
		: 'a' | 'b' | 'c' | 'd' | 'e' | 'f' | 'g' | 'h'\par 
		| 'i' | 'k' | 'l' | 'm' | 'n' | 'o' | 'p' | 'q'\par
		| 'r' | 's' | 't' | 'v' | 'x' | 'y' | 'z'\par
		;\par
		
		\textbf{decimal\_digit}\par
		: '0' | '1' | '2' | '3' | '4'\par 
		| '5' | '6' | '7' | '8' | '9'\par
		;\par
		
		\underline{\textbf{Palabras Claves}}\par
		\textbf{keyword}\par             
		: 'bool'\par
		| 'break'\par
		| 'continue'\par 
		| 'double'\par
		| 'elif'\par
		| 'else'\par
		| 'false'\par
		| 'if'\par
		| 'include'\par
		| 'int'\par
		| 'method'\par
		| 'null'\par
		| 'return'\par
		| 'string'\par
		| 'true'\par
		| 'void'\par
		| 'while'\par
		;\par
		
		\underline{\textbf{Literales}}\par
		\textbf{literal}\par
		: boolean\_literal\par
		| integer\_literal\par
		| double\_literal\par
		| string\_literal\par
		| null\_literal\par
		;\par
		
		\underline{\textbf{Literales Booleanos}}\par
		\textbf{boolean\_literal}\par
		: 'true'\par
		| 'false'\par
		;\par
		
		\underline{\textbf{Literales Enteros}}\par
		\textbf{integer\_literal}\par
		: decimal\_digit\par
		;\par
		
		\underline{\textbf{Literales flotantes}}\par
		\textbf{double\_literal}\par
		: decimal\_digit+ '.' decimal\_digit+\par
		;\par
		
		\underline{\textbf{Literales de Cadenas}}\par
		\textbf{string\_literal}\par
		: '\"' string\_literal\_character* '\"'\par
		;\par
		
		\textbf{string\_literal\_character}\par
		: '<Cualquier caracter, excepto " (U+0022)'\par
		;\par
		
		\underline{\textbf{Literales Nulos}}\par
		\textbf{null\_literal}\par
		: 'null'\par
		;\par
		
		\underline{\textbf{Operadores y signos de puntuación}}\par		
		\textbf{operator\_or\_punctuator}\par
		: '\{'  |\par
		  '\}'  |\par
	  	  '['  |\par
	  	  ']'  |\par
	  	  '('  |\par
	  	  ')'  |\par
	  	  '.'  |\par
	  	  ','  |\par       
		  ':'  |\par
		  ';'  |\par
		  '+'  |\par
		  '-'  |\par
		  '*'  |\par
		  '/'  |\par
		  '\%' |\par
		  '**' |\par  
		  '!'  |\par
		  '='  |\par
		  '<'  |\par
		  '>'  |\par
		  '\&\&' |\par
		  '||' |\par
		  '==' |\par 
		  '!=' |\par
		  '<=' |\par
		  '>=' |\par
		  '+=' |\par
		  '-=' |\par
		  '*=' |\par
		  '/=' |\par
		  '\%=' |\par 
		  '**=' |\par
		  '\&\&=' |\par
		  '||=' |\par
		  '\^='\par
		;\par
		
		\underline{\textbf{Directivas}}\par
		\textbf{directive}\par
		: 'include' file\_name ';'\par
		;\par
		
		
\end{document}

