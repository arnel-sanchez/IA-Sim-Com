Primeramente nos propusimos simular una carrera de MotoGP contrareloj, luego de que tenemos los agentes(un agente es una tupla moto-piloto) y el ambiente(un ambiente es una tupla clima-pista) generados se simula cada sección de la pista, se calcula la influencia que ejerce el ambiente sobre el agente y las propias interacciones del agente consigo mismo, el tiempo que le demora a cada piloto en circular dicha sección de la pista y la aceleración y la acción que escogerá dicho piloto en esa sección, ya sea mediante una función redefinida desd el DSL o por una Inteligencia Artificial.

\subsection{Clima}
	En cada sección de la pista se hacen pequeñas variaciones a los parámetros del clima, pero no al estado del clima, para darle un poco más de complejidad al sistema, luego de cada vuelta sí se modifica completamente el estado del clima y sus parámetros. Las simulaciones del clima se realizan mediante la distribución normal, que nos permite generar una variable aleatoria pero sin cambios tan bruscos en el promedio de sus casos.
	
\subsection{Interacciones}
	Como habíamos mencionado existen varias interacciones entre un agente y un ambiente y dentro de un propio agente:
	\begin{itemize}
		\item Si la temperatura baja se enfrían los neumáticos y se pierde adherencia al pavimento
		\begin{itemize}
			\item Disminuye el paso por curva
			\item Disminuye el paso por recta
			\item Aumenta la probabilidad de caerse de la moto
		\end{itemize} 
		\item Si la temperatura aumenta se calientan los neumáticos y se gana adherencia al pavimento, se desgastan más rápido
		\begin{itemize}
			\item Aumenta el paso por curva
			\item Aumenta el paso por recta
			\item Disminuye la probabilidad de caerse de la moto
			\item Aumenta la probabilidad de romper el motor de la moto
			\item Aumenta la probabilidad de reventar los neumáticos de la moto
		\end{itemize}
		\item Si la visibilidad baja
		\begin{itemize}
			\item Disminuye el paso por curva
			\item Disminuye el paso por recta
			\item Aumenta la probabilidad de caerse de la moto
		\end{itemize}
		\item Si la visibilidad aumenta
		\begin{itemize}
			\item Aumenta el paso por curva
			\item Aumenta el paso por recta
			\item Disminuye la probabilidad de caerse de la moto
		\end{itemize}
		\item Si la humedad aumenta se pierde adherencia al pavimento
		\begin{itemize}
			\item Disminuye el paso por curva
			\item Disminuye el paso por recta
			\item Aumenta la probabilidad de caerse de la moto
		\end{itemize}
		\item Si la humedad baja se gana adherencia al pavimento
		\begin{itemize}
			\item Aumenta el paso por curva
			\item Aumenta el paso por recta
			\item Disminuye la probabilidad de caerse de la moto
		\end{itemize}
	
		\item Si el viento de de frente
		\begin{itemize}
			\item Disminuye el paso por curva
			\item Disminuye el paso por recta
			\item Aumenta la probabilidad de reventar los neumáticos de la moto
		\end{itemize}
		\item Si el viento de de espaldas
		\begin{itemize}
			\item Aumenta el paso por curva
			\item Aumenta el paso por recta
			\item Aumenta la probabilidad de caerse de la moto
			\item Aumenta la probabilidad de romper el motor de la moto
		\end{itemize}
		\item Si el viento de de lado
		\begin{itemize}
			\item Aumenta el paso por curva
			\item Aumenta el paso por recta
			\item Aumenta la probabilidad de caerse de la moto
		\end{itemize}
	
		\item Si el estado del clima es soleado aumenta la temperatura
		\item Si el estado del clima es lluvioso aumenta la humedad
		\item Si el estado del clima es nublado condiciones perfectas para correr
		
		\item Si disminuye la rigidez del chasis tiene más torsión
		\begin{itemize}
			\item Aumenta el paso por curva
			\item Disminuye el paso por recta
		\end{itemize}
		\item Si aumenta la rigidez del chasis tiene menos torsión
		\begin{itemize}
			\item Disminuye el paso por curva
			\item Aumenta el paso por recta
		\end{itemize}
	
		\item Si disminuye la frenada se frena en más distancia
		\begin{itemize}
			\item Aumenta el paso por curva
			\item Disminuye el paso por recta
		\end{itemize}
		\item Si aumenta la frenada se frena en menos distancia
		\begin{itemize}
			\item Disminuye el paso por curva
			\item Aumenta el paso por recta
		\end{itemize}
	\end{itemize}
	De acuerdo a las interacciones anteriores se debe escoger el neumático lo más preciso posible y este provocará una combinación entre estos factores.